%%%%%%%%%%%%%%%%%%%%%%%%%%%%%%%%%%%%%%%%%
% Masters/Doctoral Thesis 
% LaTeX Template
% Version 2.5 (27/8/17)
%
% This template was downloaded from:
% http://www.LaTeXTemplates.com
%
% Version 2.x major modifications by:
% Vel (vel@latextemplates.com)
%
% This template is based on a template by:
% Steve Gunn (http://users.ecs.soton.ac.uk/srg/softwaretools/document/templates/)
% Sunil Patel (http://www.sunilpatel.co.uk/thesis-template/)
%
% Template license:
% CC BY-NC-SA 3.0 (http://creativecommons.org/licenses/by-nc-sa/3.0/)
%
%%%%%%%%%%%%%%%%%%%%%%%%%%%%%%%%%%%%%%%%%

%----------------------------------------------------------------------------------------
%	PACKAGES AND OTHER DOCUMENT CONFIGURATIONS
%----------------------------------------------------------------------------------------

\documentclass[
11pt, % The default document font size, options: 10pt, 11pt, 12pt
oneside, % Two side (alternating margins) for binding by default, uncomment to switch to one side
english, % ngerman for German
singlespacing, % Single line spacing, alternatives: onehalfspacing or doublespacing
%draft, % Uncomment to enable draft mode (no pictures, no links, overfull hboxes indicated)
%nolistspacing, % If the document is onehalfspacing or doublespacing, uncomment this to set spacing in lists to single
%liststotoc, % Uncomment to add the list of figures/tables/etc to the table of contents
%toctotoc, % Uncomment to add the main table of contents to the table of contents
%parskip, % Uncomment to add space between paragraphs
%nohyperref, % Uncomment to not load the hyperref package
headsepline, % Uncomment to get a line under the header
%chapterinoneline, % Uncomment to place the chapter title next to the number on one line
%consistentlayout, % Uncomment to change the layout of the declaration, abstract and acknowledgements pages to match the default layout
]{MastersDoctoralThesis} % The class file specifying the document structure

\usepackage[utf8]{inputenc} % Required for inputting international characters
\usepackage[T1]{fontenc} % Output font encoding for international characters

\usepackage{mathpazo} % Use the Palatino font by default

\usepackage[backend=bibtex,style=authoryear,natbib=true]{biblatex} % Use the bibtex backend with the authoryear citation style (which resembles APA)

\addbibresource{example.bib} % The filename of the bibliography

\usepackage[autostyle=true]{csquotes} % Required to generate language-dependent quotes in the bibliography

%----------------------------------------------------------------------------------------
%	MARGIN SETTINGS
%----------------------------------------------------------------------------------------

\geometry{
	paper=a4paper, % Change to letterpaper for US letter
	inner=2.5cm, % Inner margin
	outer=3.8cm, % Outer margin
	bindingoffset=.5cm, % Binding offset
	top=1.5cm, % Top margin
	bottom=1.5cm, % Bottom margin
	%showframe, % Uncomment to show how the type block is set on the page
}

%----------------------------------------------------------------------------------------
%	THESIS INFORMATION
%----------------------------------------------------------------------------------------

\thesistitle{Using Software Testing to Repair Models} % Your thesis title, this is used in the title and abstract, print it elsewhere with \ttitle
\supervisor{Dr. James \textsc{Smith}} % Your supervisor's name, this is used in the title page, print it elsewhere with \supname
\examiner{} % Your examiner's name, this is not currently used anywhere in the template, print it elsewhere with \examname
\degree{Doctor of Philosophy} % Your degree name, this is used in the title page and abstract, print it elsewhere with \degreename
\author{Marco \textsc{Radavelli}} % Your name, this is used in the title page and abstract, print it elsewhere with \authorname
\addresses{} % Your address, this is not currently used anywhere in the template, print it elsewhere with \addressname

\subject{Engineering and Applied Sciences} % Your subject area, this is not currently used anywhere in the template, print it elsewhere with \subjectname
\keywords{} % Keywords for your thesis, this is not currently used anywhere in the template, print it elsewhere with \keywordnames
\university{\href{http://www.unibg.it}{University of Bergamo}} % Your university's name and URL, this is used in the title page and abstract, print it elsewhere with \univname
\department{\href{https://www.unibg.it/ingegneria}{School of Engineering}} % Your department's name and URL, this is used in the title page and abstract, print it elsewhere with \deptname
\group{\href{https://cs.unibg.it}{Computer Science Group}} % Your research group's name and URL, this is used in the title page, print it elsewhere with \groupname
\faculty{\href{https://www.unibg.it/engineering}{Department of Management, Production and Information Engineering}} % Your faculty's name and URL, this is used in the title page and abstract, print it elsewhere with \facname

\AtBeginDocument{
\hypersetup{pdftitle=\ttitle} % Set the PDF's title to your title
\hypersetup{pdfauthor=\authorname} % Set the PDF's author to your name
\hypersetup{pdfkeywords=\keywordnames} % Set the PDF's keywords to your keywords
}



\usepackage{color}
\usepackage{amsthm}
\usepackage{stfloats}
\usepackage{paralist}
\usepackage{graphicx}
\usepackage{url}
\usepackage{booktabs}
\usepackage{subcaption}
\usepackage{amssymb}% http://ctan.org/pkg/amssymb
\usepackage{pifont}% http://ctan.org/pkg/pifont
\newcommand{\cmark}{\ding{51}}%
\newcommand{\xmark}{\ding{55}}%

\newcommand{\dom}{\ensuremath{\mathit{Dom}}\xspace}
\newcommand{\oracle}{\ensuremath{\mathit{oracle}}\xspace}
\newcommand{\oraclet}{\ensuremath{\mathit{oracle(t)}}\xspace}
\newcommand{\allTestsTrue}{\ensuremath{\mathit{allTestsTrue(c)}}\xspace}
\newcommand{\allTestsFalse}{\ensuremath{\mathit{allTestsFalse(c)}}\xspace}
\newcommand{\fcc}{\ensuremath{\mathit{fcc}}\xspace}
\newcommand{\fccs}{\ensuremath{\mathit{fccs}}\xspace}
\newcommand{\features}{\ensuremath{\mathit{features}}\xspace}
\newcommand{\overConstr}{over-constraining\xspace}
\newcommand{\underConstr}{under-constraining\xspace}
\newcommand{\true}{\ensuremath{\mathit{true}}\xspace}
\newcommand{\false}{\ensuremath{\mathit{false}}\xspace}
\newcommand{\sat}{\ensuremath{\mathit{isSAT}}\xspace}
\newcommand{\unsat}{\ensuremath{\mathit{UNSAT}}\xspace}
\newcommand{\fccSet}{\ensuremath{\mathit{FCC}}\xspace}
\newcommand{\testSuite}{\ensuremath{T}\xspace}
\newcommand{\exTestSuite}{\ensuremath{\mathit{\testSuite_e}}\xspace}

\newcommand{\m}{\ensuremath{\mathcal{M}}\xspace}
\newcommand{\mfU}{\ensuremath{\m_{\mathit{f1}}}\xspace}
\newcommand{\mfO}{\ensuremath{\m_{\mathit{f2}}}\xspace}
\newcommand{\mO}{\ensuremath{\m_o}\xspace}
\newcommand{\mRep}{\ensuremath{\m^\prime}\xspace}

\newcommand{\benchReal}{\ensuremath{\mathtt{BENCH_{REAL}}}\xspace}
\newcommand{\benchMut}{\ensuremath{\mathtt{BENCH_{MUT}}}\xspace}

\newcommand{\onlySelection}{onlySelection\xspace}
\newcommand{\atgt}{\textsf{ATGT}\xspace}
\newcommand{\espresso}{\textsf{Espresso}\xspace}
\newcommand{\jbool}{\textsf{JBool}\xspace}
\newcommand{\qm}{\textsf{QM}\xspace}

\newcommand{\exampleM}{\textsf{example}\xspace}
\newcommand{\register}{\textsf{register}\xspace}
\newcommand{\django}{\textsf{django}\xspace}
\newcommand{\tightVnc}{\textsf{tight\_vnc}\xspace}
\newcommand{\rhiscom}{\textsf{rhiscom}\xspace}
\newcommand{\erpSpl}{\textsf{ERP-SPL}\xspace}
\newcommand{\windows}{\textsf{windows}\xspace}
\newcommand{\cg}{\cellcolor{lightgray}}

%\usepackage[bookmarks=false]{hyperref}
%\hypersetup{
%colorlinks = true, % false: boxed links; true: colored links
%linkcolor=black, % color of internal links
%citecolor=black, % color of links to bibliography
%urlcolor=black % color of external links
%}

\usepackage{makecell}

%\usepackage{algorithmicx}
\usepackage{algorithm} % http://ctan.org/pkg/algorithms
\usepackage{algpseudocode} % http://ctan.org/pkg/algorithmicx
%\usepackage{algorithmic}
%\algsetup{linenosize=\small}
\algrenewcommand\algorithmicindent{0.6em}
\makeatletter
\renewcommand{\ALG@beginalgorithmic}{\footnotesize}
\makeatother

\usepackage{bibentry}

\makeatletter
%%%%%%%%%%%%%%%%%%%%%%%%%%%%%% Textclass specific LaTeX commands.
\theoremstyle{plain}
\newtheorem{thm}{\protect\theoremname}
\theoremstyle{definition}
\newtheorem{defn}{\protect\definitionname}
\theoremstyle{remark}
\newtheorem{example}{\protect\examplename}
\theoremstyle{remark}
\newtheorem{observation}{\protect\observationname}
\theoremstyle{plain}
\newtheorem{assumption}{\protect\assumptionname}
\theoremstyle{plain}
\newtheorem{corollary}{Corollary}

\providecommand{\definitionname}{Definition}
\providecommand{\examplename}{Example}
\providecommand{\observationname}{Observation}
\providecommand{\theoremname}{Theorem}
\providecommand{\assumptionname}{Assumption}

\newcommand{\red}[1]{\textcolor{red}{#1}}

\newenvironment{nscenter} % https://tex.stackexchange.com/questions/98839/center-and-centering-both-add-space-above-centerline-does-not
{\parskip=0pt\par\nopagebreak\centering}
{\par\noindent\ignorespacesafterend}

\usepackage{xcolor,colortbl}
\definecolor{light-gray}{gray}{0.8}

\usepackage{multirow}

\usepackage{xspace}

\usepackage{centernot}
%\usepackage[numbers,super]{natbib}

\usepackage{listings}
\lstset{
	columns=fullflexible,
	showstringspaces=false,
	numberbychapter=false,
	captionpos=b
}
\renewcommand{\lstlistingname}{Code}% Listing -> Code

\newcounter{researchquestionCount}
\newcommand{\researchquestion}[1]{\stepcounter{researchquestionCount}\begin{itemize}\item [\textbf{RQ\arabic{researchquestionCount}:}] \emph{#1}\end{itemize}}

\usepackage[colorinlistoftodos,prependcaption,textsize=normalsize,backgroundcolor=blue!10]{todonotes}
%\presetkeys{todonotes}{inline}{}
\usepackage{comment}
\usepackage{tikz}

\usepackage{bm}


\begin{document}

\frontmatter % Use roman page numbering style (i, ii, iii, iv...) for the pre-content pages

\pagestyle{plain} % Default to the plain heading style until the thesis style is called for the body content

%----------------------------------------------------------------------------------------
%	TITLE PAGE
%----------------------------------------------------------------------------------------

\begin{titlepage}
\begin{center}

\vspace*{.06\textheight}
{\scshape\LARGE \univname\par}\vspace{1.5cm} % University name
\textsc{\Large Doctoral Thesis}\\[0.5cm] % Thesis type

\HRule \\[0.4cm] % Horizontal line
{\huge \bfseries \ttitle\par}\vspace{0.4cm} % Thesis title
\HRule \\[1.5cm] % Horizontal line
 
\begin{minipage}[t]{0.4\textwidth}
\begin{flushleft} \large
\emph{Author:}\\
\href{http://www.johnsmith.com}{\authorname} % Author name - remove the \href bracket to remove the link
\end{flushleft}
\end{minipage}
\begin{minipage}[t]{0.4\textwidth}
\begin{flushright} \large
\emph{Supervisor:} \\
\href{http://www.jamessmith.com}{\supname} % Supervisor name - remove the \href bracket to remove the link  
\end{flushright}
\end{minipage}\\[3cm]
 
\vfill

\large \textit{A thesis submitted in fulfillment of the requirements\\ for the degree of \degreename}\\[0.3cm] % University requirement text
\textit{in the}\\[0.4cm]
\groupname\\\deptname\\[2cm] % Research group name and department name
 
\vfill

{\large \today}\\[4cm] % Date
%\includegraphics{Logo} % University/department logo - uncomment to place it
 
\vfill
\end{center}
\end{titlepage}

%----------------------------------------------------------------------------------------
%	DECLARATION PAGE
%----------------------------------------------------------------------------------------

\begin{declaration}
\addchaptertocentry{\authorshipname} % Add the declaration to the table of contents
\noindent I, \authorname, declare that this thesis titled, \enquote{\ttitle} and the work presented in it are my own. I confirm that:

\begin{itemize} 
\item This work was done wholly or mainly while in candidature for a research degree at this University.
\item Where any part of this thesis has previously been submitted for a degree or any other qualification at this University or any other institution, this has been clearly stated.
\item Where I have consulted the published work of others, this is always clearly attributed.
\item Where I have quoted from the work of others, the source is always given. With the exception of such quotations, this thesis is entirely my own work.
\item I have acknowledged all main sources of help.
\item Where the thesis is based on work done by myself jointly with others, I have made clear exactly what was done by others and what I have contributed myself.\\
\end{itemize}
 
\noindent Signed:\\
\rule[0.5em]{25em}{0.5pt} % This prints a line for the signature
 
\noindent Date:\\
\rule[0.5em]{25em}{0.5pt} % This prints a line to write the date
\end{declaration}

\cleardoublepage

%----------------------------------------------------------------------------------------
%	QUOTATION PAGE
%----------------------------------------------------------------------------------------

\vspace*{0.2\textheight}

\noindent\enquote{\itshape Thanks to my solid academic training, today I can write hundreds of words on virtually any topic without possessing a shred of information, which is how I got a good job in journalism.}\bigbreak

\hfill Dave Barry

%----------------------------------------------------------------------------------------
%	ABSTRACT PAGE
%----------------------------------------------------------------------------------------

\begin{abstract}
\addchaptertocentry{\abstractname} % Add the abstract to the table of contents
%The Thesis Abstract is written here (and usually kept to just this page). The page is kept centered vertically so can expand into the blank space above the title too\ldots
Software testing is an important phase in the software development process, aiming at locating faults in artifacts, in order to achieve a degree of confidence that the software behaves according to a specification.
While most of the techniques in software testing are applied to debugging, fault-localization, and repair of code, to the best of our knowledge there are fewer works regarding the application of software testing to locating faults in models and to the automated repair of such faults.
The goal of this PhD project proposal is to study how testing can be applied to repair models. %of configurable systems. %, using software testing to obtain a set of failing test cases to drive the repair process.
We describe the research approach and discuss the application cases of combinatorial and feature models.
We then discuss future work of applying testing to repair models for other scenarios, such as timed automata.
\end{abstract}

%----------------------------------------------------------------------------------------
%	ACKNOWLEDGEMENTS
%----------------------------------------------------------------------------------------

\begin{acknowledgements}
\addchaptertocentry{\acknowledgementname} % Add the acknowledgements to the table of contents
The acknowledgments and the people to thank go here, don't forget to include your project advisor\ldots
\end{acknowledgements}

%----------------------------------------------------------------------------------------
%	LIST OF CONTENTS/FIGURES/TABLES PAGES
%----------------------------------------------------------------------------------------

\tableofcontents % Prints the main table of contents

\listoffigures % Prints the list of figures

\listoftables % Prints the list of tables

%----------------------------------------------------------------------------------------
%	ABBREVIATIONS
%----------------------------------------------------------------------------------------

\begin{abbreviations}{ll} % Include a list of abbreviations (a table of two columns)

\textbf{SPL} & \textbf{S}oftware \textbf{P}roduct \textbf{L}ine(s)\\
\textbf{CIT} & \textbf{C}ombinatorial \textbf{I}nteraction \textbf{T}esting\\

\end{abbreviations}

%----------------------------------------------------------------------------------------
%	PHYSICAL CONSTANTS/OTHER DEFINITIONS
%----------------------------------------------------------------------------------------

%\begin{constants}{lr@{${}={}$}l} % The list of physical constants is a three column table

% The \SI{}{} command is provided by the siunitx package, see its documentation for instructions on how to use it

%Speed of Light & $c_{0}$ & \SI{2.99792458e8}{\meter\per\second} (exact)\\
%Constant Name & $Symbol$ & $Constant Value$ with units\\

%\end{constants}

%----------------------------------------------------------------------------------------
%	SYMBOLS
%----------------------------------------------------------------------------------------

%\begin{symbols}{lll} % Include a list of Symbols (a three column table)

%$a$ & distance & \si{\meter} \\
%$P$ & power & \si{\watt} (\si{\joule\per\second}) \\
%Symbol & Name & Unit \\

%\addlinespace % Gap to separate the Roman symbols from the Greek

%$\omega$ & angular frequency & \si{\radian} \\

%\end{symbols}

%----------------------------------------------------------------------------------------
%	DEDICATION
%----------------------------------------------------------------------------------------

%\dedicatory{For/Dedicated to/To my\ldots} 

%----------------------------------------------------------------------------------------
%	THESIS CONTENT - CHAPTERS
%----------------------------------------------------------------------------------------

\mainmatter % Begin numeric (1,2,3...) page numbering

\pagestyle{thesis} % Return the page headers back to the "thesis" style

% Include the chapters of the thesis as separate files from the Chapters folder
% Uncomment the lines as you write the chapters

%\include{Chapters/Chapter1}
%\include{Chapters/Chapter2} 
%\include{Chapters/Chapter3}
%\include{Chapters/Chapter4} 
%\include{Chapters/Chapter5} 

\chapter{Introduction}

\section{Software Testing}
\subsection{Black-box software testing}

\section{Motivation}
%Rationale
Traditional testing techniques provide test generation, fault localization and program repair. %, but \textit{model} repair is  % of \textit{code} artifacts. 
If we assume a \textit{fault} to be a discrepancy between a model (\textit{problem space}) and the system implementation (\textit{solution space}) for a particular test case, we can distinguish two scenarios: (1) the model is correct and the implementation is faulty, and (2) the implementation is correct and the model is faulty.
While the most common case is that the model reflects the specification, and the error resides in the system implementation, this is not always the case.
The specification may change throughout the software system life cycle, and it may happen that the implementation is updated but the model(s) remain(s) outdated, leading to inconsistencies that reduce system maintainability \cite{rost2013software}. %that increase the effort for project stakeholders to understand the system, and for software engineers to modify it, in case new functionalities shall be added, or an update is needed. 
Such inconsistencies are often due to budget constraints, as updating models is a difficult and error-prone activity to perform manually. % In such cases, only the system code is updated as it is the only artifact that strictly needs to be modified to meet the new requirements.
In other cases, the model is derived from the software implementation, often even long time after the program is put in operation. 
Moreover, with the increase of complexity of software systems, also their models are becoming larger: the Linux Kernel, for instance, has currently more than 13,000 features (in release 3.9) \cite{passos2018study}. %, describing only partial aspect of the system, at a specific level of abstraction. 
%\todo{aclune volte perchè il modello è derivato dal sw, magari anche dopo.}
The goal of this research proposal is to study how software testing techniques can be applied to drive automatic repair of software models.
Model \textit{repair} can be useful whenever the model is outdated w.r.t. the implementation, or a defined oracle or the specification.
Sometimes an engineer also wants to detect which is a model of a current aspect of an implemented system, and can use testing techniques not only to find bugs in a system, but also to \textit{repair} or to entirely \textit{learn} a model from the system implementation. 
Outcomes of this project may have an impact in reducing the effort of software maintenance, which is one of the costliest phases in software life cycle \cite{zarnekow2005distribution}.

\section{Research Questions and Objective}



\chapter{State of the Art}
There are several works on automated program repair \cite{nguyen_semfix:_2013,le_goues_systematic_2012,ARCURI20113494}. However, to the best of our knowledge, very few studies address the problem of repairing the \textit{model} artifacts, instead of the implementation code, and the problem of repair \textit{automatically} existing models of software systems w.r.t. the implementation is still an open issue. 
In model-driven engineering, there are approaches aiming at repairing inconsistencies between models \cite{Macedo:2013:MRT:3107656.3107752}; Tran et al. show a way to address the problem \textit{manually} for the Linux Kernel \cite{Tran:1999:FRR:781995.782007}; Nadi et al. present a method to automatically \textit{mine} conditions under which a system behaves in a certain way \cite{NadiBKC14}, and there are methods to statistically infer constraints from data~\cite{chiang_unified_2011,abukwaik_extracting_2016}, but they are not directly applicable to repair existing models made of sets of constraints and they do not guarantee complete accuracy. 
%\todo{to describe why there is repair only for code and not for models}.
A quality-based model refactoring framework assessing quality of merging operations among SPL models, expressed in UML~\cite{rubin_quality_2013}, supports maintainability of models describing relations among features. It represents an approach to model repair, although it is specific to one kind of models.
The need of a fault-driven constraint repair process was already envisioned in~\cite{henard_towards_2013}, but any empirical experiment were yet performed.

\section{Software Testing}
\subsection{Combinatorial Testing}
\subsection{The Oracle Problem}

\section{Manipulation Techniques}
\subsection{SAT and SMT solvers}
\subsection{Search-based Software Engineering}
\subsection{Model Checkers}

\section{Automated Program Repair}

\section{Model Transformations}
\subsection{Software Product Lines}
\subsection{Timed Automata}
\subsection{Abstract State Machines}

\section{Systematic Literature Review - Model Repair}



\chapter{Approach and Uniqueness}
%The research approach consists of two stages: the reviewing stage, and the formulation stage.
%In the reviewing stage, we evaluate existing techniques that may be useful for the objective, and explore possible application scenarios, to determine the kind of models that can be repaired from software testing results.
%In the formulation stage, for each scenario, we \textit{decline} the repair problem to that particular kind of model, and apply or tailor existing techniques to propose a solution to the model repair problem.

We devise an iterative process to automatically repair models: the model is modified until all the non-conformances between the model and the system (revealed by the generated tests), are solved. Fig. \ref{fig:approach} shows an overview of such test-driven repair approach. 
Among testing techniques to be employed, black-box testing is an effective technique to detect faults in a system focusing on the inputs, without needing access to the code, but requiring simply an oracle that states if a particular test %(i.e., a system configuration)
passes or fails. 
Combinatorial Interaction Testing (CIT), in particular, has shown to achieve high coverage in software systems for which a parametric, configurable model can be defined. Moreover, we use mutation analysis and search-based methods for \textit{feature models}, a model that has also a graphical tree representation, and these methods tend to apply \textit{small} edits to the model, helping preserve domain knowledge.
%Benchmarks should be selected for evaluation, and tools to implement the repair process should be reused if existing, or customized or built if needed.

Unlike processes that detect faults and repair code, %the proposed repair process targets models and is based on the assumption that the fault resides in the model, and the implementation is correct. %, the rationale that drives this PhD proposal is the investigation of software testing techniques to automatically \textit{repair} models.
the main hypothesis of this project is that testing techniques can be used not only for fault detection and localization, but also for model repair.
This methodology aims at changing the model \textit{with less impact as possible} (to preserve domain knowledge) to \textit{repair} the non-conformance w.r.t. the \textit{solution space} (it can be the current implementation, a new specification, or an available \textit{oracle}).
The applied repairs are local and little, based on the \textit{competent modeler} assumption, that the model is \textit{a little} outdated, and minimal changes are enough. 

\begin{figure}[!tb]
	\centering
	\includegraphics[width=.8\columnwidth]{images/repair.png}
	\caption{Test-Driven process to repair models}
	\label{fig:approach}
\end{figure}

%We plan to evaluate effectiveness and efficiency of the test-driven repair techniques on different applications, with real-world models and systems: from literature and from industrial cases. Apart from the already investigated applications to combinatorial and feature models, we plan to study a scenario of repair of 
%timed automata (TA), high order logic expressions, and finite state machines, for example. % they all have
%Timed automata are mainly used to describe the behavior of communication protocols, real-time systems and cyber-physical systems.
%The model could be abstracted to obtain parameters representing timing constraints in the guards or in the locations, resulting in a Parametric Timed Automata (PTA), and tests drive the detection of a configuration for such parameters, that reflects the \textit{updated} system.
%Each test, in this case, could be an \textit{execution trace}.%, that can happen or not in a given system, or according to the new specification.

%The dissemination plan consists in applying test-driven repair to real-world software systems, and possibly release tools, in addition to publishing papers to venues related to software testing or to the particular application for which test-driven repair has been applied.
%If real-world case studies coming from collaborations with industries in the sector are found, it would be an added value.


\chapter{Request-Driven Repair}
Feature models are a widely used modeling notation for variability management in software product line (SPL) engineering. 
%In order to keep an SPL and its feature model aligned, feature models must be changed by including/excluding new features and products, either because faults in the model are found or to reflect the normal evolution of the SPL.
%Such changes can be complex and error-prone due to the size of the feature model.
%Therefore, 
We developed an approach that \textit{repairs} a feature model w.r.t. a given update request in the form of combinations representing a set of configurations to be accepted or rejected, that may be detected by \textit{failing} test cases, or directly by engineer domain knowledge. % from a change in the specification.
The method is based on an evolutionary algorithm that iteratively mutates the original feature models and checks if the update request is semantically fulfilled.
We employ mutations such as switching an optional feature to mandatory, or changing an \emph{or} group to an \emph{and} group, based on \cite{arcaini2018evolutionary}.
% \cite{arcaini2018evolutionary,ARCAINI201964}.
We generate faults between two real versions of feature models of the {\tt MobileMedia}, {\tt HelpSystem},  {\tt SmartHome}, and {\tt ERP\_SPL} systems in the SPLOT repository\footnote{\url{http://52.32.1.180:8080/SPLOT/feature_model_repository.html}}, and we notice that although our approach does not guarantee to completely update all the possible feature models, on average, around 89\% of requested changes are applied, with minimal edits.%, helping in preserving domain knowledge.

%\subsection{Test generation and fault localization}
%The process is driven by update configurations or failure-inducing combinations in the input parameters of the system, and black-box (in particular, combinatorial) test generation and fault detection and localization techniques, are important to support engineers in determining such inputs for the repair process.
%To this purpose, we make use of a tool to make it easier the editing of combinatorial models, and the test suite generation, called CTWedge \cite{IWCTGargantini2018}, and we devise a fault localization process, based on combinatorial testing, that guarantees to find the correct detection of failure-inducing combinations, under the assumption that the maximum strength of the real failure-inducing combinations is known.

\chapter{Failure-Driven Repair}
\section{Repair of Configuration Constraints}
A model for a combinatorial problem consists of parameters which can take various domain values. Combinatorial models may have also constraints among parameter values to, for example, model inconsistencies between certain hardware components, limitations of the possible system configurations, or simply because of design choices \cite{gargantini_combinatorial_2017}.
%Some methods have been introduced to automate the process of inferring constraints, but they do not aim at \textit{repairing} existing ones \cite{abukwaik_extracting_2016,Temple:2016:UML:2934466.2934472}. 
%Therefore, in \cite{gargantini_combinatorial_2017}
The devised iterative approach uses a fault-localization tool based on combinatorial testing, called BEN \cite{ghandehari2018combinatorial}, and CIT policies introduced in \cite{Gargantini16:validation} to find failure-inducing combinations of parameter values.
The model is then repaired \textit{logically}, by translating such failure-inducing combinations into expressions in propositional logic.
The repairs are of two types, depending on whether the model is true and the system (i.e., the oracle) false for a given test case (in this case, the model is \textit{under-constrained}) or vice-versa (in this case, the model is \textit{over-constrained}). Tab. \ref{table:testsandfaults} reports possible scenarios in which such condition may occur in a system with three boolean parameters A, B, C, that map to directives in a C program in which both B and C can be enabled only if also A is activated.
\begin{table}[!tb]
	\caption{Test suites with faults (in gray)}\label{table:testsandfaults}
	\begin{comment}
	\begin{tabular}{c|c|c||c|c|c}
	A & B & C & oracle & \mfU & \mfO\\
	& & & $A \rightarrow B$,$ A \rightarrow C$ & $A \rightarrow B$ & $A \rightarrow B$,$C$\\
	\hline 
	T & T & T & T & T & T \\
	T & T & F & F & \cg T & F \\
	T & F & T & F & F & F \\
	F & T & T & T & T & T \\% & \underConstr fault\\
	F & T & F & T & T & \cg F\\
	F & F & T & T & T & T\\
	F & F & F & T & T & \cg F\\
	\end{tabular}
	\end{comment}
	\resizebox{.8\columnwidth}{!}{
		\begin{subtable}[t]{.5\columnwidth}
			\centering
			\caption{\underConstr fault}
			\label{table:underConstrFault}
			\begin{tabular}{c|c|c||c|c}
				A & B & C & \mfU & oracle\\
				\hline 
				T & T & F & \cg T & \cg F\\% & \underConstr fault\\
				T & F & F & F & F\\
			\end{tabular}
		\end{subtable}%
		\begin{subtable}[t]{.5\columnwidth}
			\centering
			\caption{\overConstr fault}
			\label{table:overConstrFault}
			\begin{tabular}{c|c|c||c|c}
				A & B & C & \mfO & oracle\\
				\hline 
				F & T & T & T & T\\
				F & F & T & T & T\\
				F & T & F & \cg F & \cg T\\% & \overConstr fault\\
				F & F & F & \cg F & \cg T\\% & \overConstr fault\\
			\end{tabular}
		\end{subtable}%
	}
\end{table}


Experiments for five real-world systems (Libssh, Telecom, Aircraft, Concurrency, and Django) show that our approach can repair on average 37\% of conformance faults. Moreover, we also notice that it can infer and repair parameter constraints for the configurations that lead to a successful startup of Django, a well-known open source web application framework written in Python.

%For test generation we used a framework called CTWedge

%\subsubsection{XSS Vulnerabilities Constraints}
%Combinatorial models are also suitable to model parts of XSS attack vectors, with a limited number of possible values.
%We applied our iterative approach also to such models, using as oracle a script that runs the web application under test, inject the attack vector in a particular input field in a form, and checks wether the injected Javascript code has been present in the response page.  We evaluated our approach empirically on four systems from the Web Application Vulnerability Scanner Evaluation Project (WAVSEP), and on two real-world web applications, obtaining an accurate model of the constraints among parameters that cause an attack vector to break the sanitization function of the system.

\section{Test-Driven Model Evolution in XSS Vulnerability Detection}
\section{Repairing Timed Automata Clock Guards through Abstraction and Testing}


\chapter{Tools}
\section{CTWedge: Migrating Combinatorial Interaction Test Modeling and Generation to the Web}
\section{MixTgTe: Efficient and Guaranteed Detection of t-Way Failure-Inducing Combinations}

\chapter{Conclusions and Future Work}
We illustrated the research project of using software testing techniques to drive the inference and repair of models of software systems.
It is a novel application of software testing that goes beyond detecting and localizing faults in code, and that performs little repairs to preserves domain knowledge and support engineers in maintaining consistency between all the software artifacts, and localizing faults also in the model. 
We presented applications to combinatorial and feature models. As future work, we plan to apply test-driven repair of other model types, namely timed automata, and abstract state machines. Furthermore, we plan to improve the process of combinatorial models repair by improving the accuracy of the fault localization strategy.%, for instance with MixTgTe 

%\section{Acknowledgment}
%\todo{}
%I would like to thank my supervisor Angelo Gargantini for his advice, support and help in my research, and also thank the other collaborators in the research activity so far, for their precious contribution.

%\chapter{Bibliography}

 

%----------------------------------------------------------------------------------------
%	THESIS CONTENT - APPENDICES
%----------------------------------------------------------------------------------------

%\appendix % Cue to tell LaTeX that the following "chapters" are Appendices

% Include the appendices of the thesis as separate files from the Appendices folder
% Uncomment the lines as you write the Appendices

%\include{Appendices/AppendixA}
%\include{Appendices/AppendixB}
%\include{Appendices/AppendixC}

%----------------------------------------------------------------------------------------
%	BIBLIOGRAPHY
%----------------------------------------------------------------------------------------

\printbibliography[heading=bibintoc]

%----------------------------------------------------------------------------------------

\end{document}  
